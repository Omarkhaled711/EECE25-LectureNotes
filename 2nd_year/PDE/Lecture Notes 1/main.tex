\documentclass[12 pt]{article}
\usepackage[utf8]{inputenc}
\usepackage{a4}
\usepackage{indentfirst}
\usepackage{graphicx}
\usepackage{float}
\usepackage{tabularx}
\usepackage{multicol}
\usepackage{tikz, adjustbox}
\usepackage[most]{tcolorbox}
\usepackage{wrapfig}
\usepackage{amssymb}
\usepackage{amsmath}
\usepackage{hyperref}
\newcolumntype{M}[1]{>{\centering\arraybackslash}m{#1}}
\usepackage{titlesec}
\usepackage{listings}
\usepackage{esint}
\usepackage{xcolor}
\usepackage{empheq} 
\usepackage{tcolorbox}
\setlength\parindent{0pt}
\definecolor{codegreen}{rgb}{0,0.6,0}
\definecolor{codegray}{rgb}{0.5,0.5,0.5}
\definecolor{codepurple}{rgb}{0.58,0,0.82}
\definecolor{backcolour}{rgb}{0.95,0.95,0.92}
\emergencystretch=1em
\lstdefinestyle{mystyle}{
    backgroundcolor=\color{backcolour},  
    commentstyle=\color{codegreen},
    keywordstyle=\color{magenta},
    numberstyle=\tiny\color{codegray},
    stringstyle=\color{codepurple},
    basicstyle=\footnotesize\ttfamily,
    breakatwhitespace=false,         
    breaklines=true,                 
    captionpos=b,                    
    keepspaces=true,                 
    numbers=left,                    
    numbersep=5pt,                  
    showspaces=false,                
    showstringspaces=false,
    showtabs=false,                  
    tabsize=2,
    frame=single,
}
\lstset{style=mystyle}
% Yellow:
\definecolor{BgYellow}{HTML}{FFF59C}
\definecolor{FrameYellow}{HTML}{F7A600}
% Pink:
\definecolor{BgPink}{HTML}{EF6FA7}
\definecolor{FramePink}{HTML}{E5446E}


% Yellow Sticky Note (YStkyNote):
\newtcolorbox{YStkyNote}[1][]{%
    enhanced,
    before skip=2mm,after skip=2mm, 
    width=0.6\textwidth, % width of the sticky note
    boxrule=0.2mm,
    colback=BgYellow, colframe=FrameYellow, % Colors
    attach boxed title to top left={xshift=0cm,yshift*=0mm-\tcboxedtitleheight},
    varwidth boxed title*=-3cm,
    % The titlebox:
    boxed title style={frame code={%
        \path[left color=FrameYellow,right color=FrameYellow,
        middle color=FrameYellow]
        ([xshift=-0mm]frame.north west) -- ([xshift=0mm]frame.north east)
        [rounded corners=0mm]-- ([xshift=0mm,yshift=0mm]frame.north east)
        -- (frame.south east) -- (frame.south west)
        -- ([xshift=0mm,yshift=0mm]frame.north west)
        [sharp corners]-- cycle;
        },interior engine=empty,
    },
    sharp corners,rounded corners=southeast,arc is angular,arc=3mm,
    % The "folded paper" in the bottom right corner:
    underlay={%
        \path[fill=BgYellow!80!black] ([yshift=3mm]interior.south east)--++(-0.4,-0.1)--++(0.1,-0.2);
        \path[draw=FrameYellow,shorten <=-0.05mm,shorten >=-0.05mm,color=FrameYellow] ([yshift=3mm]interior.south east)--++(-0.4,-0.1)--++(0.1,-0.2);
        },
    drop fuzzy shadow, % Shadow
    fonttitle=\bfseries, 
    title={#1}
}
% Pink Sticky Note (PStkyNote):
\newtcolorbox{PStkyNote}[1][]{%
    enhanced,
    before skip=2mm,after skip=2mm, 
    width=0.4\textwidth, % width of the sticky note
    boxrule=0.2mm, 
    colback=BgPink, colframe=FramePink, % Colors
    attach boxed title to top left={xshift=0cm,yshift*=0mm-\tcboxedtitleheight},
    varwidth boxed title*=-3cm,
    % The titlebox:
    boxed title style={frame code={%
        \path[left color=FramePink,right color=FramePink,
        middle color=FramePink]
        ([xshift=-0mm]frame.north west) -- ([xshift=0mm]frame.north east)
        [rounded corners=0mm]-- ([xshift=0mm,yshift=0mm]frame.north east)
        -- (frame.south east) -- (frame.south west)
        -- ([xshift=0mm,yshift=0mm]frame.north west)
        [sharp corners]-- cycle;
        },interior engine=empty,
    },
    sharp corners,rounded corners=southeast,arc is angular,arc=3mm,
    % The "folded paper" in the bottom right corner:
    underlay={%
        \path[fill=BgPink!80!black] ([yshift=3mm]interior.south east)--++(-0.4,-0.1)--++(0.1,-0.2);
        \path[draw=FramePink,shorten <=-0.05mm,shorten >=-0.05mm,color=FramePink] ([yshift=3mm]interior.south east)--++(-0.4,-0.1)--++(0.1,-0.2);
        },
    drop fuzzy shadow, % Shadow
    fonttitle=\bfseries, 
    title={#1}
}

\title{\huge{Lecture Notes1: PDE}}
\author{}
\date{}
\begin{document}
\maketitle
\tableofcontents
\newpage
\section{Introduction}
\subsection{Mathematical model}
 The process of describing a real world problem in mathematical terms
 \subsection{differential equations}
 An equation that relates one or more unknown functions and their derivatives.
 \begin{itemize}
     \item \textbf{ODE:} An equation that contains one or several derivatives of unknown function and single variable.
     \item \textbf{PDE:} An equation that contains partial derivatives of unknown function of tow or more variables 
 \end{itemize}
 \subsection{Common examples of PDEs}
 \begin{itemize}
     \item $u_t +cu_x =0$
     \item $u_{xx} + u_{yy} =f(x,y)$
     \item $a(x,y)u_{xx}+2u{xy}+3x^2u_{yy}=4e^x$
     \item $u_xu_{xx}+(u_y)^2=0$
     \item $(u_{xx})^2+u_{yy}+a(x,y)u_x+b(x,y)u=0$
 \end{itemize}
 \section{Basic concepts and definitions}
 \subsection{Definition 1}
 The order of a PDE is the order of the highest order derivative in the equation.
 \subsubsection*{Examples}
 \begin{itemize}
     \item \textbf{first order}\begin{enumerate}
         \item $u_t +c u_x=0$
     \end{enumerate}
         
     \item \textbf{second order}\begin{enumerate}
         \item $u_{xx}+u_{yy}=f(x,y)$
         \item $u_xu_{xx}+(u{y})^2=0$
     \end{enumerate}
 \end{itemize}
\subsection{Definition 2}
A PDE is linear if it's linear in the unknown function and all its derivatives with coefficients depending only on the independent variables.
\subsubsection*{Examples}
\begin{itemize}
     \item \textbf{linear}\begin{enumerate}
         \item $u_t +c u_x=0$
          \item $u_{xx}+u_{yy}=f(x,y)$
          \item $a(x,y)u_{xx}+2u{xy}+3x^2u_{yy}=4e^x$
     \end{enumerate}
     \item \textbf{Non-linear}\begin{enumerate}
         \item $u_xu_{xx}+(u{y})^2=0$
         \item $(u_{xx})^2+u_{yy}+a(x,y)u_x+b(x,y)u=0$
     \end{enumerate}
 \end{itemize}
 \subsection{Definition 3}
 A PDE is homogeneous if the equation does not contain a term independent of the unknown function
and its derivatives.
\subsubsection*{Examples}
\begin{itemize}
     \item \textbf{linear}\begin{enumerate}
         \item $u_t +c u_x=0$
         \item $u_xu_{xx}+(u{y})^2=0$
         \item $(u_{xx})^2+u_{yy}+a(x,y)u_x+b(x,y)u=0$
     \end{enumerate}
     \item \textbf{Non-linear}\begin{enumerate}
          \item $u_{xx}+u_{yy}=f(x,y)$
          \item $a(x,y)u_{xx}+2u{xy}+3x^2u_{yy}=4e^x$
     \end{enumerate}
 \end{itemize}
 \newpage
\section{Basic exercises}
\subsection*{Example 1}
Show that $\mathbf{u(x,y) = F(xy) + xG(\frac{x}{y})}$ is a general solution of the PDE\\ $\mathbf{x^2u_{xx}-y^2_{yy}=0}$
\begin{tcolorbox}
[width=\linewidth, sharp corners=all, colback=white!95!black]
let $\alpha=xy$ and $\beta=\frac{y}{x}$

$\alpha_x=y$\hspace{0.3cm} $\alpha_y=x$\hspace{0.3cm}$\beta_x=\frac{-y}{x^2}$\hspace{0.3cm} $\beta_y=\frac{1}{x}$

$\therefore u=F(\alpha)+xG(\beta)$

$\therefore u_x= y\dot{F}+ G(\beta)- \dot{G} \frac{y}{x}$

$\therefore u_{xx} = y^2\ddot{F}+\frac{y^2}{x^3}\ddot{G}$

$\therefore u_y=x\dot{F}+\dot{G}$

$\therefore u_{yy}=x^2\dot{F}+\frac{1}{x}\ddot{G}$

$$\boxed{
\therefore x^2u_{xx}-y^2_{yy} = x^2y^2\ddot{F}+\frac{y^2}{x}\ddot{G}-x^2y^2\ddot{F}-\frac{y^2}{x}\ddot{G}=0}$$
\end{tcolorbox}
\subsection*{Example 2: PDE can have many solutions}
Find three possible solutions to any 2D Laplace Equation under specific initial
conditions and boundary conditions $\mathbf{u_{xx}+u_{yy}=0}$
\begin{tcolorbox}
[width=\linewidth, sharp corners=all, colback=white!95!black]
\textbf{first sol: }$u(x,y)=x^2-y^2$

\textbf{second sol: }$u(x,y)=ln(x^2+y^2)$

\textbf{second sol: } $e^{x}cos(y)$
\end{tcolorbox}
\section{Fundamental Theorem (Superposition)}
Let D be linear differential operators (in the variables
$x_1, x_2, . . . , x_n$), let f1 and f2 be functions (in the same variables),
and let c1 and c2 be constants.
\begin{itemize}
    \item If $u_1$ solves the linear PDE $Du = f_1$ and $u_2$ solves $Du = f_2$,
then $u = c_1u_1 + c_2u_2$ solves $Du = c_1f_1 + c_2f_2$. In particular, if
$u_1$ and $u_2$ both solve the same homogeneous linear PDE, so
does $u = c_1u_1 + c_2u_2$.
    \item If $u_1$ satisfies the linear boundary condition $Du|_A=f_1|_A$ and $u_2$ satisfies $Du|_A = f_2|_A$
, then $u = c_1u_1 + c_2u_2$ satisfies $Du|_A = c_1f_1 + c_2f_2|_A $. In particular, if $u_1$ and $u_2$ both satisfy
the same homogeneous linear boundary conditions, so does $u = c_1u_1 + c_2u_2$.
\end{itemize}
\section{Classification of linear 2nd order PDEs}
\begin{itemize}
    \item Linear Second order PDEs are used to model many systems in many different fields
    \item The general form of a linear 2nd order PDE in two variables x and y is given by: $\mathbf{Au_{xx} + Bu_{xy} + Cu_{yy} + Du_x + Eu_y + Fu = G}$ , Where A, B, C,D, E, F, and G depend only on x and y.
    \item A discriminant at some point is defined as: $\mathbf{\Delta=B^2 -4AC}$
\end{itemize}
\subsection*{The PDE is classified as follows:}
\begin{itemize}
    \item $\mathbf{\Delta>0}$: Hyperbolic
    \item $\mathbf{\Delta=0}$: Parabolic
    \item $\mathbf{\Delta<0}$: Elliptic
\end{itemize}
\end{document}
