\documentclass[12 pt]{article}
\usepackage[utf8]{inputenc}
\usepackage{a4}
\usepackage{indentfirst}
\usepackage{graphicx}
\usepackage{float}
\usepackage{tabularx}
\usepackage{multicol}
\usepackage{tikz, adjustbox}
\usepackage[most]{tcolorbox}
\usepackage{wrapfig}
\usepackage{amssymb}
\usepackage{hyperref}
\newcolumntype{M}[1]{>{\centering\arraybackslash}m{#1}}
\usepackage{titlesec}
\usepackage{listings}
\usepackage{xcolor}
\definecolor{codegreen}{rgb}{0,0.6,0}
\definecolor{codegray}{rgb}{0.5,0.5,0.5}
\definecolor{codepurple}{rgb}{0.58,0,0.82}
\definecolor{backcolour}{rgb}{0.95,0.95,0.92}
\emergencystretch=1em
\lstdefinestyle{mystyle}{
    backgroundcolor=\color{backcolour},  
    commentstyle=\color{codegreen},
    keywordstyle=\color{magenta},
    numberstyle=\tiny\color{codegray},
    stringstyle=\color{codepurple},
    basicstyle=\footnotesize\ttfamily,
    breakatwhitespace=false,         
    breaklines=true,                 
    captionpos=b,                    
    keepspaces=true,                 
    numbers=left,                    
    numbersep=5pt,                  
    showspaces=false,                
    showstringspaces=false,
    showtabs=false,                  
    tabsize=2,
    frame=single,
}
\lstset{style=mystyle}
% Yellow:
\definecolor{BgYellow}{HTML}{FFF59C}
\definecolor{FrameYellow}{HTML}{F7A600}
% Pink:
\definecolor{BgPink}{HTML}{EF6FA7}
\definecolor{FramePink}{HTML}{E5446E}


% Yellow Sticky Note (YStkyNote):
\newtcolorbox{YStkyNote}[1][]{%
    enhanced,
    before skip=2mm,after skip=2mm, 
    width=0.6\textwidth, % width of the sticky note
    boxrule=0.2mm,
    colback=BgYellow, colframe=FrameYellow, % Colors
    attach boxed title to top left={xshift=0cm,yshift*=0mm-\tcboxedtitleheight},
    varwidth boxed title*=-3cm,
    % The titlebox:
    boxed title style={frame code={%
        \path[left color=FrameYellow,right color=FrameYellow,
        middle color=FrameYellow]
        ([xshift=-0mm]frame.north west) -- ([xshift=0mm]frame.north east)
        [rounded corners=0mm]-- ([xshift=0mm,yshift=0mm]frame.north east)
        -- (frame.south east) -- (frame.south west)
        -- ([xshift=0mm,yshift=0mm]frame.north west)
        [sharp corners]-- cycle;
        },interior engine=empty,
    },
    sharp corners,rounded corners=southeast,arc is angular,arc=3mm,
    % The "folded paper" in the bottom right corner:
    underlay={%
        \path[fill=BgYellow!80!black] ([yshift=3mm]interior.south east)--++(-0.4,-0.1)--++(0.1,-0.2);
        \path[draw=FrameYellow,shorten <=-0.05mm,shorten >=-0.05mm,color=FrameYellow] ([yshift=3mm]interior.south east)--++(-0.4,-0.1)--++(0.1,-0.2);
        },
    drop fuzzy shadow, % Shadow
    fonttitle=\bfseries, 
    title={#1}
}
% Pink Sticky Note (PStkyNote):
\newtcolorbox{PStkyNote}[1][]{%
    enhanced,
    before skip=2mm,after skip=2mm, 
    width=0.4\textwidth, % width of the sticky note
    boxrule=0.2mm, 
    colback=BgPink, colframe=FramePink, % Colors
    attach boxed title to top left={xshift=0cm,yshift*=0mm-\tcboxedtitleheight},
    varwidth boxed title*=-3cm,
    % The titlebox:
    boxed title style={frame code={%
        \path[left color=FramePink,right color=FramePink,
        middle color=FramePink]
        ([xshift=-0mm]frame.north west) -- ([xshift=0mm]frame.north east)
        [rounded corners=0mm]-- ([xshift=0mm,yshift=0mm]frame.north east)
        -- (frame.south east) -- (frame.south west)
        -- ([xshift=0mm,yshift=0mm]frame.north west)
        [sharp corners]-- cycle;
        },interior engine=empty,
    },
    sharp corners,rounded corners=southeast,arc is angular,arc=3mm,
    % The "folded paper" in the bottom right corner:
    underlay={%
        \path[fill=BgPink!80!black] ([yshift=3mm]interior.south east)--++(-0.4,-0.1)--++(0.1,-0.2);
        \path[draw=FramePink,shorten <=-0.05mm,shorten >=-0.05mm,color=FramePink] ([yshift=3mm]interior.south east)--++(-0.4,-0.1)--++(0.1,-0.2);
        },
    drop fuzzy shadow, % Shadow
    fonttitle=\bfseries, 
    title={#1}
}
\title{\huge{Lecture Notes 1: Memory Types}}
\author{}
\date{}
\begin{document}
\maketitle
\tableofcontents
\newpage
\section{Memory classifications by access type}
\begin{itemize}
    \item Read-Only Memory (ROM):A memory storage device, whose contents can be read and accessed but cannot be modified.
    \item Read-Write Memory (RWM): A memory that can be read from and written to (e.g: RAM: Random Access Memory)
\end{itemize}
\section{Read-Only Memory (ROM)} 
\textbf{It's a non volatile memory.}
\begin{itemize}
    \item MPROM (Mask-programmed ROM): \begin{itemize}
        \item It's programmed at the factory
        \item Program (or data) isn't erasable
        \item The oldest type of ROM
        \item Ideal for high volume, low cost production
    \end{itemize}
    \item PROM(Programmable ROM):\begin{itemize}
        \item It can be programmed by the user only once
        \item The circuit uses high voltage to permanently eliminate inner links
        \item Bipolar transistor is typically used in theses devices, so it's fast and uses relatively high power.
    \end{itemize}
    \item EPROM(Erasable-Programmable ROM): \begin{itemize}
        \item Read-Mostly
        \item It can be programmed more than one time by applying voltage
        \item Data is erased by UV Light
        \item floating-gate MOSFET  is used in theses devices
        \item you can only erase all the data.
        \item It's not erased in place.
    \end{itemize}
    \item EEPROM(Electrically Erasable PROM):\begin{itemize}
        \item Read-Mostly
        \item Can be erased and reprogrammed by electrical signals.
        \item Floating-gate MOSFET is used in theses devices.
        \item Data is erased on byte level
        \item It can be erased in place.
    \end{itemize}
    \item Flash: \begin{itemize}
        \item Read-Mostly
        \item special type of EEPROM
        \item Faster than EEPROM
        \item Programmed in larger blocks
        \item Data is erased on block level
    \end{itemize}
\end{itemize}
\subsection{How floating-gate MOSFET works}
Flash memory works by adding (charging) or removing (discharging) electrons to and from a floating gate. A bit's 0 or 1 state depends on whether the floating gate is charged or uncharged. When electrons are present on the floating gate, current can't flow through the transistor and the bit state is 0. When a bit is programmed, this is the normal state for a floating gate transistor. When electrons are removed from the floating gate, the current is allowed to flow and the bit state is 1.

\section{Random Access Memory (RAM)}
\textbf{It's a volatile memory.}
\begin{table}[H]
\begin{tabular}{|l|l|l|}
\hline
 & \begin{tabular}[c]{@{}l@{}}SRAM\\ (Static RAM)\end{tabular} & \begin{tabular}[c]{@{}l@{}}DRAM\\ (Dynamic RAM)\end{tabular} \\ \hline
Basic cell          & flip flop (6 transistors) & capacitor (1 transistor) \\ \hline
Speed               & very fast                 & fast                     \\ \hline
Power consumtion    & consumes more power       & consumes less power      \\ \hline
Cost                & expensive                 & cheap                    \\ \hline
Size                & larger                    & smaller                  \\ \hline
refreshment circuit & doesn't need one          & needs one                \\ \hline
interface           & simple                    & more complex             \\ \hline
\end{tabular}
\end{table}
\end{document}
